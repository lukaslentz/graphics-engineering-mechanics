%Start
%
%
%******************************************************************************************************%
%                                        Penduulum                                                     %
%******************************************************************************************************%								 
% Version 1,                                                                                           %
% 07.07.2023                                                                                           %
% L.Lentz@umwelt-campus.de                                                                             %
%******************************************************************************************************%
%
%
\documentclass[tikz,border=10pt]{standalone}
\usepackage{amsmath}
\usetikzlibrary{calc,intersections,backgrounds}
%
\begin{document}
%
\tikzset{
points/.style = {circle,draw,fill=white,minimum size=0.6cm,inner sep=0pt, outer sep=0pt},
caseI/.style = {color=orange!90,line width=1.5pt},
caseII/.style = {color=purple!90,line width=1.5pt},
caseIII/.style = {color=black!40!green,line width=1.5pt},
}
%
\begin{tikzpicture}
%
\def\l{3};% length penduulum
\def\ang{-60};% angle of rotation 
\def\r{0.2};% radius mass
%
%
\coordinate (O) at (0,0);% Origin
\coordinate (M) at ({cos(\ang)*\l},{sin(\ang)*\l});% end of y-axis
%
\draw[](O)--(M);
\draw[fill = gray, draw = gray] (M) circle (\r) node[anchor = ]{$M$};
%
\end{tikzpicture}
%
\end{document}
%
%
%Ende