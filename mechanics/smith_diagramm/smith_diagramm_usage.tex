%Start
%
%
%******************************************************************************************************%
%                                 Smith diagramm construction                                          %
%******************************************************************************************************%
% Illustrate the construction of smith's diagramm from                      													 %
%******************************************************************************************************%
% Version 1,                                                                                           %
% 07.07.2023                                                                                           %
% L.Lentz@umwelt-campus.de                                                                             %
%******************************************************************************************************%
%
%
\documentclass[tikz,border=10pt]{standalone}
\usepackage{amsmath}
\usetikzlibrary{calc,intersections,backgrounds}
%
\newcommand*{\rom}[1]{\expandafter\@slowromancap\romannumeral #1@}
%
\begin{document}
%
\tikzset{
points/.style = {circle,draw,fill=white,minimum size=0.6cm,inner sep=0pt, outer sep=0pt},
caseI/.style = {color=orange!90,line width=1.5pt},
caseII/.style = {color=purple!90,line width=1.5pt},
caseIII/.style = {color=black!40!green,line width=1.5pt},
}
%
\begin{tikzpicture}
%
%	Define stress values
%'''''''''''''''''''''''''''''''''''''''''''''''''''''''''''
\def\scal{1/100}; %scal factor for stress in diagramm
\def\sigw{300*\scal};
\def\sigs{450*\scal};
\def\sigsx{0.5*\sigs};
\def\re{520*\scal};
\def\sm{140*\scal};
\def\sa{100*\scal};
%'''''''''''''''''''''''''''''''''''''''''''''''''''''''''''
%
\def\os{0.1};% small offset
\def\los{1};% large offset
\def\tlw{0.7pt};%	thinn-lines width
\def\mlw{1.pt};% medium-lines width
\def\lw{1.7pt};% thick-lines width
%
\clip[](-\los,-\sigw-\los) rectangle (\re+2*\los,\re+2*\los);% clip canvas because of large construction lines
%
\coordinate (O) at (0,0);% Origin
\coordinate (Y) at (0,\re+\los);% end of y-axis
\coordinate (X) at (\re+\los,0);% end of x-axis
%
\coordinate (A) at (0,\sigw);
\coordinate (B) at (\sigsx,\sigs);
\coordinate (C) at (\re,\re);
\coordinate (D) at (\sigsx,0);
\coordinate (E) at (0,-\sigw);
%
\coordinate (S) at (\sm,\sm+\sa);
%
%coordinate system
\draw[-latex,line width = \mlw](O)--(X) node[at end, right]{$\sigma_m$};
\draw[-latex,line width = \mlw]($(O)+(0,-\sigw-\los)$)--(Y) node[at end, above]{$\sigma$};
%
%kappa-lines
\draw[draw=none,name path=horizon](Y)--+(\re+2*\los,0);
\draw[draw=none,name path=kappa1](O)--($(O)!50!(C)$);
\draw[draw=none,name path=kappa](O)--($(O)!50!(S)$);
\path[name intersections={of=horizon and kappa1,by=K1}];
\path[name intersections={of=horizon and kappa,by=K}];
\draw[](O)--(K1)node[at end, anchor = south]{$\kappa=1$};
\draw[](O)--(K)node[at end, anchor = south]{$\kappa_{\text{ist}}$};
%
%
\draw[draw=none,name path=line1]($(C-|O)+(-\os,0)$)--(C);
\draw[line width=\tlw](O)--+(-\os,0)node[at end, left]{$0$};
%
\draw[draw=none,name path=line2](A)--($(A)!+50!(B)$);
\path[name intersections={of=line1 and line2,by=H1}];
%
\draw[draw=none,name path=line3](H1)--+(0,-\re);
\draw[draw=none,name path=line4](E)--($(E)!50!(D)$);
\path[name intersections={of=line3 and line4, by=H2}];
%
\begin{scope}[on background layer]
\draw[draw=none,name path=gline]($(H2)+(0.75*\los,0)$)--($($(H2)+(0.75*\los,0)$)!50!(0.75*\los,-\sigw)$);
\draw[draw=none,name path=yaxis](Y)--+(0,-5*\re);
\path[name intersections={of=gline and yaxis, by=G}];
\clip[](0,\re+0.5*\los)--(\re+0.75*\los,\re+0.5*\los)--($(H2)+(0.75*\los,0)$)--(G)--cycle;
\draw[step=.5,gray,thin] (0,-\sigw-\los) grid (\re+\los,\re+\los);
\end{scope}
%
\draw[draw=none,name path=line5](A)--(H1)--(C);
\draw[draw=none,name path=line6](S)--+(0,\re);
\draw[draw=none,name path=line7](S)--($(S)!50!(2*\sm,2*\sm+\sa)$);
%
\path[name intersections={of=line5 and line6, by=H3}];
\path[name intersections={of=line5 and kappa, by=H4}];
\path[name intersections={of=line5 and line7, by=H5}];
%
%
\draw[dashed,line width=\tlw]($(S-|O)+(-\os,0)$)--(S)node[at start,left]{$\sigma_o$};
\draw[dashed,line width=\tlw]($(S|-O)+(0,-\os)$)--(S)node[at start,below]{$\sigma_m$};
%
\draw[caseI](H3)--(S);
\draw[caseII](H4)--(S);
\draw[caseIII](H5)--(S);
%
\draw(H3-|O)--+(-\os,0)node[at end,left,caseI]{$\sigma_{1}$};
\draw(H4-|O)--+(-\os,0)node[at end,left,caseII]{$\sigma_{2}$};
\draw(H5-|O)--+(-\os,0)node[at end,left,caseIII]{$\sigma_{3}$};
%
\draw[fill=black,draw=none] (S) circle (2pt);
%
\draw[color = blue,line width = \lw](A)--(H1)--(C)--(H2)--(E);
%
\begin{scope}[yshift=-2cm,xshift=2cm]
\draw[caseI](0,0)--+(1,0)node[at end,right]{$\text{Fall 1:}~\sigma_m=\text{konstant}$};
\draw[caseII](0,-0.4)--+(1,0)node[at end,right]{$\text{Fall 2:}~\kappa=\text{konstant}$};
\draw[caseIII](0,-0.8)--+(1,0)node[at end,right]{$\text{Fall 3:}~\sigma_a=\text{konstant}$};
\end{scope}
%
\end{tikzpicture}
%
\end{document}
%
%
%
%End